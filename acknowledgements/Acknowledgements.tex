\section*{Ringraziamenti}

Innanzitutto ci tengo a ringraziare il professor Stefano Carrazza per avermi accolto nella collaborazione Qibo, 
per avermi proposto un progetto di tesi molto stimolante e per la straordinaria esperienza al CERN.\\
Inoltre ci tengo a ringraziare il professor Carrazza per i numerosi suggerimenti che mi ha fornito durante tutto 
il lavoro della mia tesi.\\
Durante quest’anno mi sono sentito molto felice di esser parte della comunità di Qibo, perché mi sono sentito 
parte di una collaborazione dinamica, ricca di stimoli e colma di persone preparatissime.
Per queste ragioni mi sono subito fidato del progetto di tesi che mi è stato proposto perché ero certo di 
essere in buone mani.\\
L’esperienza al CERN è stata tanto inaspettata quanto apprezzata. Ho avuto l’occasione unica di parlare 
e conoscere giovani aspiranti fisici provenienti da tutto il mondo; ho avuto l’occasione di seguire le 
lezioni di professori di fama internazionale, che hanno ulteriormente accresciuto la mia passione per la Fisica.\\
Ci tengo anche a ringraziare Matteo che è sempre stato disponibile durante tutto quest’anno, che mi ha 
aiutato a dare una una forma ed una struttura alle mie idee e a farle diventare una tesi della quale sono 
soddisfatto.\\
Ci tengo anche a ringraziarlo per il supporto, per i consigli e per il tempo passato insieme al CERN questa estate.\\

Ci tengo a ringraziare la mia famiglia, nella cui unità non smetto mai di credere.
Ringrazio i miei genitori per avermi dato una educazione non convenzionale.\\

Ci tengo a ringraziare i miei amici che sono la parte più bella della mia vita. Sto scrivendo questo cappello 
introduttivo, dopo avervi ringraziato singolarmente e solo ora che mi sono preso del tempo per riflettere 
sul rapporto che ho con ciascuno di voi mi rendo conto di quanto bene vi voglia e di quanto siate importanti per me. 
Negli ultimi anni lo studio mi ha catturato e spesso ho avuto la sensazione di trascurare alcune delle mie amicizie, 
ma voglio ringraziarvi per la pazienza e la comprensione che avete avuto.\\

Ringrazio Lisbo per le fantastiche cene, per le partite del Milan e per le storie divertenti che ci hai raccontato.\\
Ringrazio Ripo per le foto che anno immortalato tutte le nostre vacanze, per le cene in via Lario, per le storie 
dei tuoi studenti e per quella giornata divertente a casa dei tuoi parenti in Abruzzo.\\
Ringrazio Bigno per essere il collante del nostro gruppo (non vedo l’ora di esplorare Pavia insieme!). 
Senza di te il gruppo non sarebbe unito e amalgamato.\\
Ringrazio Zolo per le sue buffe smorfie e per la sua compagnia silenziosa. \\
Ringrazio Senne e Viktor per la nostra amicizia che resiste la distanza. \\
Ringrazio Anna per le domande divertenti con le quali ci intrattiene in vacanza ogni anno e per aver fondato 
lo “Sleepover Club”, del quale sono fiero di esser parte.\\
Ringrazio la Maddi per avermi mostrato scampoli di Bologna, per volermi bene e per il tempo passato insieme 
durante le nostre vacanze estive.\\
Ringrazio Pelle per la nostra condivisa passione per il cocco, per la cucina alla Wellington, e per tutte le 
serate passate insieme a casa di Mambo.\\
Ti ringrazio anche per le tue battute sagaci e pungenti, e per tutte le cose di geopolitica che mi hai insegnato.\\
Ringrazio Alice per le serate a Ginevra, nelle quali mi sono divertito molto. Sono molto contento che sei 
diventata parte del nostro gruppo di amici, la tua solarità è apprezzata da tutti.\\
Ringrazio Alice, Alia e Lauren per le gite, le serate e il tempo passato insieme questa estate.
In particolare ci tengo a ringraziare Alia per le nostre serate sotto la pioggia e per le nostre conversazioni, 
che sono state una finestra su un mondo che non conoscevo.\\
Ringrazio Ammar per le lunghe e interessanti conversazioni, per avermi insegnato moltissimo sulla sua cultura e 
per avermi fatto divertire con le sue storie.\\
Ringrazio Fidaa per avermi raccontato una parte del mondo che non conoscevo, per essersi presa a cuore tutti i 
miei problemi, per la sua vitalità contagiosa.
Fidaa sei una persona con un animo buono e coraggioso, e non vedo l’ora di scoprire dove la tua resilienza e 
determinazione ti porteranno.\\
Ringrazio Dosha per essere un cavallo pazzo. Ti ringrazio per tutte le nostre conversazioni stimolanti, per 
avermi accolto nei Galli Cedroni e per le vacanze trascorse insieme.\\
Ringrazio Giulio per essere stato un amico e un confidente, per le sue storie intriganti, per avermi introdotto 
al mondo dei banchetti.\\
Ringrazio Tommi per le partite di calcetto, per i numerosissimi passaggi in auto, ma soprattutto per averci 
ospitato per tre mesi indimenticabili a Baceno.
Porterò quell’esperienza sempre nel cuore.\\
Ringrazio Chiara S. per volermi sempre molto bene e per i mesi trascorsi a Baceno.\\
Ringrazio Chiara D. per le nostre conversazioni e per avermi supportato durante questi anni.\\
Ringrazio Chiara M. per le nostre conversazioni introspettive e per essere sempre presente e interessata alla mia 
vita nonostante la distanza che ci separa.\\
Ringrazio Dave per essere stato al mio fianco durante tutto il percorso universitario, per avermi supportato 
e motivato anche nei momenti in cui vacillavo. Ti ringrazio anche per avermi fatto assaggiare cibi stranissimi, 
che in nessun altro luogo del mondo potrei trovare. Tutti i bellissimi momenti che abbiamo passato insieme in 
questi anni rimarranno sempre impressi nella mia memoria, fra questi ne menziono solo alcuni: le partitelle a 
Plesios, la grigliata a Plesios, la vacanza a Casasco, le giornate sulla neve, le finali NBA …\\
Ringrazio Luca per essere (insieme a Bigno) il collante del nostro gruppo. La tua presenza è fondamentale per 
il benessere di tutti durante le nostre uscite, durante le nostre vacanze. Vederti e parlare con te mi mette 
sempre di buon umore. Non dimenticherò mai la quantità di zucchero che mettevi nel te’ a Baceno, non dimenticherò 
mai lo scherzo che ti abbiamo fatto in Toscana. Ti ringrazio Luca per sopportare tutte le mie domande pungenti di 
storia e geopolitica, sei per me un pozzo senza fondo di informazioni interessantissime.\\
Ringrazio Edo per tutte le nostre conversazioni, che sono state un vero arricchimento e occasione di crescita.
Parlare con Edo è sempre un piacere, perché ogni frase di uno si incastra con una frase dell’altro componendo un 
nuovo e inaspettato ragionamento, che ci porta sempre a scoprire e a comprendere qualcosa di nuovo sul mondo che ci circonda. 
Ti ringrazio anche per tutti gli elogi con i quali mi hai aiutato ad abbattere le mie insicurezze.
Il tempo che abbiamo passato insieme in questi anni è indimenticabile e importantissimo per me: Napoli, le 
mattinate in Bicocca, la vacanza silenziosa a Varzo, …\\
Grazie davvero Edo.\\
Ringrazio Pleba per la nostra longeva amicizia. La nostra amicizia è indissolubile e non riesco ad immaginarmi 
una vita nella quale non siamo amici. Ti ringrazio per volermi bene, per supportarmi, per motivarmi e stimolarmi 
da più di 10 anni. Ti ringrazio per essere stato un amico vero sin dall’inizio del liceo. Nessuno ha seguito la 
mia evoluzione e la mia crescita con il tuo interesse e la tua vicinanza. Hai sempre avuto a cuore tutte le cose 
che mi hanno riguardato nel corso degli anni come se ti riguardassero direttamente.
Hai vissuto le mie vittorie come fossero tue e per questo ti sarò sempre grato. Ripercorrere tutto quello che 
abbiamo vissuto insieme sarebbe impossibile, ma ci tengo a menzionare qualche episodio divertente: i votacci in 
condotta, le grigliate e pizzate a Varazze, i tornei di calcio di fine anno, le finali NBA, i superbowl, il 
fantacalcio, le recensioni di tutti i bar di Milano, la gita a Berlino, la gita ad Assisi, …\\
Ringrazio Mambo per la nostra amicizia indissolubile. Sin dal liceo sei stato per me un esempio da seguire e un 
modello a cui ispirarmi.
Sei un punto di riferimento. Prima gli Scriddy, poi il gruppo del Milan e infine i Cavers: i nostri gruppi di 
amici cambiano nome, ma la nostra amicizia è una costante. Potrei scrivere una lista lunghissima di episodi 
indimenticabili che abbiamo vissuto insieme, ma mi limiterò a citarne qualcuno: ascoltare la musica di Frah 
Quintale su un letto matrimoniale in vacanza, i Galli Cedroni, il rito di iniziazione Scriddy, le partite a 
Risiko mangiando i pancake della Tere, le serate a vedere il Milan, …
Non potrò mai ringraziarti abbastanza 
per il supporto incondizionato	che mi hai dato quest’anno. Questi mesi che ho trascorso a casa tua sono stati 
sicuramente fra i più belli del mio percorso universitario. \\
Grazie Mambo. \\

Dopo aver ringraziato i miei amici ci tengo a ringraziare la persona più importante della mia vita. 
Faccio fatica a trovare le parole per 
esprimere l’amore e la gratitudine che provo nei tuoi confronti.
In questi anni il nostro rapporto è evoluto e maturato in modo naturale, come il fluire di un fiume, e oggi, 
a pochi giorni dal trasloco nella nostra prima casa, mi emoziona ripensare al giorno in cui ci siamo conosciuti.
Ricordo ancora nitidamente quella serata al planetario. Sei stata l’imprevisto più bello di questi due anni di 
studi magistrali.
Ti ringrazio per volermi bene, per sopportarmi e supportarmi. 
Non so come avrei fatto senza di te in questi anni.
Grazie Cecilia.\\

Infine ci tengo a ringraziare me stesso.\\ 
Questi anni sono stati impegnativi e troppo spesso ho screditato i miei successi.
Voglio elogiare la mia resilienza, la mia costanza, la mia tenacia e la mia determinazione.
Voglio ringraziare il giovane Niccolò, che è riuscito a superare in modo eccellente ogni esame, nonostante le 
attanaglianti insicurezze.\\
Durante questi anni ho affrontato ogni esame con lo scopo di voler dimostrare agli altri e a 
me stesso di essere in grado di superare qualsiasi tipo di ostacolo si presentasse nel mio cammino.
La paura di fallire mi ha ostacolato e non mi ha permesso di ragionare e studiare nel pieno delle mie possibilità.
D'ora in avanti questo paradigma cambierà, non ho più nulla da dimostrare a nessuno.\\
Il paradigma che muoverà le mie scelte sarà quello della curiosità e del piacere di imparare.\\

Per ultima ci tengo a ringraziare la Fisica, per aver dato un senso alle mie giornate negli ultimi sei anni.\\
Mi sono iscritto a questo percorso di studi senza particolari motivazioni, se non un vago e mal definito sentimento 
di rivalsa nei confronti delle materie che avevano maggiormente ostacolato i miei studi liceali.\\
Oggi e per sempre sarò grato a quella scelta poco ponderata.\\

