
\chapter{A novel data encoding technique}

The paper \cite{P_rez_Salinas_2020} is a groundbreaking paper in quantum machine learning.\\
Significant effort by the quantum machine learning community has been dedicated to deepening our understanding of 
quantum re-uploading models, leading to a growing body of literature investigating them.\\
As we studied this paper, some natural questions emerged and we decided to investigate them by designing
a novel data encoding technique for quantum re-uploading models, whose main goal is to classify 
\textit{high-dimensional} and \textit{structured} datasets\footnote[1]{The term \textit{structured} is 
used to indicate that the components of each datapoint in the dataset may be highly correlated.}.\\
In the following sections we will discuss the questions emerged from reading the original data re-uploading
paper and a new quantum machine learning algorithm.

\section{Questions on data re-uploading}

The groundbreaking paper on the data re-uploading technique \cite{P_rez_Salinas_2020} does not address some crucial 
question, which could be an interesting starting point for future research:

\begin{itemize}
    \item\label{question:first} Is this encoding technique effective when dealing with high-dimensional datasets?\\
    The investigation conducted by Salinas et al focused on 2D, 3D and 4D data, hence this raises the natural
    question if this encoding method can deal with datasets of higher dimension.
    \item\label{question:second} Is this encoding technique effective when dealing with structured dataset (as images)?\\
    The original paper tested the data re-uploading encoding method to distinguish if a certain point was 
    inside or outside a certain geometrical boundary (circle, sphere, annulus, ...).
    However it did not investigate its performance on datasets, whose each datapoint's components could 
    be correlated to each other.
    Therefore, this raises another natural question which is if the data re-uplaoding technique can handle datasets
    like images, whose pixels are highly correlated to the sorrounding pixels.
\end{itemize}

We tried to answer to these questions by designing a new quantum machine learning algorithm based on the
data re-uploading technique.\\
I will discuss this new algorithm in the following section.

\section{Block re-uploading}

This new algorithm, which we call \textit{block re-uploading}, is inspired by classical convolutional neural 
networks and its main goal is to classify images.
Images are composed of pixels, and each pixel is typically represented as a vector of three 
values corresponding to the intensity of the primary colors: red, green, and blue (RGB). 
Each of these values typically ranges from 0 to 255 in an 8-bit color depth image, which is common in 
digital images.\\
It is essential to observe that not all the information contained in an image is essential for
classification purposes.
Indeed, many preprocessing techniques, such as PCA, have been used in the literature to reduce 
the amount of information fed into a (quantum) neural network.\\
to preserve information redundancy in the images and maintain the full dimensionality of the dataset. 
This approach aligns with our objective of addressing question \ref{question:first}.\\

The fundamental idea behind the \textit{block re-uploading} algorithm is based on the observation that 
neighboring pixels in an image are highly correlated. 
Consequently, we chose to divide each image into blocks and upload each block onto a separate qubit\footnote[1]{We wanted to make these blocks as similar to each other as possible, but dimensional constraints 
prevented them from being identical. For instance, an $8\times8$ image cannot be divided in three
identical blocks.} (figure \ref{fig:block}).\\

\begin{figure}[h]
    \centering
    \begin{subfigure}[b]{0.45\textwidth}
        \centering
        \includegraphics[scale=0.65]{sections/chapters/Chapter6/Images/Blocks-Circuit/blocks.pdf}
    \end{subfigure}
    \begin{subfigure}[b]{0.45\textwidth}
        \centering
        \includegraphics[scale=0.75]{sections/chapters/Chapter6/Images/Blocks-Circuit/circuit.pdf}
    \end{subfigure}
    \caption{An image with even width and height can be divided into exactly four equal blocks. 
    Each block will then be encoded onto a separate qubit. As we will explain in the next paragraph the 
    \textit{block re-uploading} architecture has three components per layer: embedding circuit, 
    entanglement circuit, pooling circuit.}
    \label{fig:block}
\end{figure}

Apart from answering to the questions \ref{question:second}, our main goal was to 
investigate the \textit{Depth vs Width trade-off}.
The \textit{Depth vs Width trade-off} can be more explicitly referred to as the
\textit{Sequential vs Parallel uploading trade-off}.
If we do not split the image into blocks, the entire image will be encoded into a single qubit, 
resulting in sequential distribution of information across the quantum circuit. 
In contrast, if we divide the image into blocks and encode each block onto a separate qubit, 
the information will be distributed both sequentially and in parallel. As the number of blocks increases, 
the distribution of information becomes progressively more parallel, reaching the limit where each 
block consists of a single pixel.
This trade-off raises the natural questions: \textit{which distribution is better? Sequential or parallel?}.
As is common when evaluating trade-offs, the optimal solution often lies between the two extremes. 
Therefore, we predict that the best approach to distributing information across the quantum circuit 
will involve a balanced mix of sequential and parallel encoding.\\

\begin{figure}
    \centering
    \begin{subfigure}[b]{0.3\textwidth}
        \centering
        \includegraphics[width=\textwidth]{sections/chapters/Chapter6/Images/Sequential-Parallel/sequential_blocks.pdf}
        \label{fig:sequential-block}
    \end{subfigure}
    \hfill
    \begin{subfigure}[b]{0.3\textwidth}
        \centering
        \includegraphics[width=\textwidth]{sections/chapters/Chapter6/Images/Sequential-Parallel/proportionate_blocks.pdf}
        \label{fig:prop-block}
    \end{subfigure}
    \hfill
    \begin{subfigure}[b]{0.3\textwidth}
        \centering
        \includegraphics[width=\textwidth]{sections/chapters/Chapter6/Images/Sequential-Parallel/parallel_blocks.pdf}
        \label{fig:parallel-block}
    \end{subfigure}
    \hfill
    \begin{subfigure}[b]{0.3\textwidth}
        \centering
        \includegraphics[width=\textwidth]{sections/chapters/Chapter6/Images/Sequential-Parallel/sequential.pdf}
        \label{fig:sequential-circ}
    \end{subfigure}
    \hfill
    \begin{subfigure}[b]{0.3\textwidth}
        \centering
        \includegraphics[width=\textwidth]{sections/chapters/Chapter6/Images/Sequential-Parallel/proportionate.pdf}
        \label{fig:prop-circ}
    \end{subfigure}
    \hfill
    \begin{subfigure}[b]{0.3\textwidth}
        \centering
        \includegraphics[width=\textwidth]{sections/chapters/Chapter6/Images/Sequential-Parallel/parallel.pdf}
        \label{fig:parallel-circ}
    \end{subfigure}
    \caption{This figure shows three possible splitting of a $4\times4$ image and the corresponding 
    circuit. 
    If the image is not divided into blocks, it will be uploaded to a single qubit, 
    representing a fully-sequential approach. 
    Dividing the image into 4 blocks results in a 4-qubit architecture, 
    positioning it roughly at the midpoint of the sequential-parallel spectrum.
    If the image is divided into 16 blocks, each block will contain only 1 pixel. 
    This scenario exemplifies the fully-parallel uploading limit, as each pixel is assigned to 
    a separate qubit.}
    \label{fig:three-splitting}
\end{figure}


The \textit{block re-uploading} algorithm is a layered algorithm and each layer has three components: 
an embedding circuit, an entanglement structure and a pooling circuit (figure \ref{fig:components}).

\begin{figure}[h]
    \centering
    \includegraphics[scale=0.65]{sections/chapters/Chapter6/Images/Two-Layers-A/two_layer.pdf}
    \caption{A four qubits and two layers \textit{block re-uploading} architecture. Each layer has three 
    components: an embedding circuit, an entanglement structure and a pooling circuit.
    Layers are separated by another entanglement structure.}
    \label{fig:components}
\end{figure}

\paragraph{Embedding}
The embedding component is responsible to encode each block of an image onto a different qubit of 
the quantum circuit (figure \ref{fig:block-feature}).\\
For instance, an $8\times8$ image can be divided in 4 identical $4\times4$ blocks. 
Therefore, each $4\times4$ block is a 16 dimensional vector which requires 
$\left\lceil \frac{d}{3} \right\rceil = \left\lceil \frac{16}{3} \right\rceil = 6$ unitary matrices 
to be encoded onto a qubit.
In particular, each block needs 16 rotation gates, one per pixel.
Therefore, the unitary matrices necessary to encode a block $\bm{x} = (x_1, x_2, ..., x_{16})$ will be:

\begin{align}
    U^1_1(\bm{\phi_1}) &= U_1(\bm{x}, \bm{\theta_1}) = R_Z(\phi_{1,1}) R_Y(\phi_{1,2}) R_Z(\phi_{1,3}) \\
    U^1_2(\bm{\phi_2}) &= U_2(\bm{x}, \bm{\theta_2}) = R_Z(\phi_{2,1}) R_Y(\phi_{2,2}) R_Z(\phi_{2,3}) \\
    &\vdots \\
    U^1_6(\bm{\phi_6}) &= U_1(\bm{x}, \bm{\theta_6}) = R_Z(\phi_{6,1}) R_Y(\phi_{6,2}) R_Z(\phi_{6,3}) 
\end{align}

where $\bm{\theta_i} = (w_{i,1}, w_{i,2}, w_{i,3}, b_{i,1})$.
The angles will be defined as a linear combination of pixels and weights, for example $\bm{\phi_1}$:

\begin{align}
    \phi_{1,1} &= x_1 \cdot w_{1,1} + b_{1,1} \\
    \phi_{1,2} &= x_2 \cdot w_{1,2} + b_{1,2} \\
    \phi_{1,3} &= x_3 \cdot w_{1,3} + b_{1,3} \\
\end{align}


\begin{figure}[h]
    \centering
    \begin{quantikz}
        &&& \gate{U^1_1} & \gate{U^1_2} & \gate{U^1_3} & \gate{U^1_4} & \gate{U^1_5} & \gate{U^1_6} &&& \\
        &&& \gate{U^2_1} & \gate{U^2_2} & \gate{U^2_3} & \gate{U^2_4} & \gate{U^2_5} & \gate{U^2_6} &&& \\
        &&& \gate{U^3_1} & \gate{U^3_2} & \gate{U^3_3} & \gate{U^3_4} & \gate{U^3_5} & \gate{U^3_6} &&& \\
        &&& \gate{U^4_1} & \gate{U^4_2} & \gate{U^4_3} & \gate{U^4_4} & \gate{U^4_5} & \gate{U^4_6} &&& \\
    \end{quantikz}
    \caption{Embedding circuit for an $8\times8$ image divided in 4 identical $4\times4$ blocks.}
    \label{fig:block-feature}
\end{figure}


\paragraph{Entanglement structure}
Since each block is correlated with its neighboring blocks, we decided to use an entanglement 
structure in which entangling gates create connections between adjacent blocks (figure 
\ref{fig:entanglement-feature}).
We chose CZ, as the only entangling gate.
Therefore, the entanglement structure aims to ensure that each qubit shares information only with 
qubits that contain related information.

\begin{figure}[h]
    \centering
    \begin{quantikz}
        &&& \ctrl{1} & \ctrl{2}   &       &     &&&&\\
        &&& \control{}  &    &       &   \ctrl{2}    &     &&&\\
        &&&    &        \control{}    &       &   & \ctrl{1}   &&&\\
        &&&    &     &    &       \control{}   &   \control{}  &&&\\
    \end{quantikz}
    \caption{Entanglement structure for an image divided in 4 blocks. The first qubit will 
    communicate with the second one and the third one, as the first block is adjacent to the second and the third one.
    Then, the second block will be connected to the fourth one and the third to fourth one.}
    \label{fig:entanglement-feature}
\end{figure}

\paragraph{Pooling}
In classical machine learning the pooling layers are used to make the network less sensitive to small translations 
and distortions in the input data.\\
Therefore, we decided to mimic the pooling component of classical convolutional neural networks, by 
adding an X rotation gate per qubit, whose angle is defined as the linear combination of the max (or average) value 
of a block and weights (figure \ref{fig:pooling-feature}).
Therefore, if we consider again an $8\times8$ image divided in 4 identical $4\times4$ blocks: 

\begin{align}
    \bm{x_1} &= (x_{1,1}, x_{1,2}, ..., x_{1,16}) 
    \qquad
    \rightarrow
    \qquad
    max(\bm{x_1}) \\
    \bm{x_2} &= (x_{2,1}, x_{2,2}, ..., x_{2,16}) 
    \qquad
    \rightarrow
    \qquad
    max(\bm{x_2}) \\
    \bm{x_3} &= (x_{3,1}, x_{3,2}, ..., x_{3,16}) 
    \qquad
    \rightarrow
    \qquad
    max(\bm{x_3}) \\
    \bm{x_4} &= (x_{4,1}, x_{4,2}, ..., x_{4,16}) 
    \qquad
    \rightarrow
    \qquad
    max(\bm{x_4}) \\
\end{align}

the angles of the four X rotation gates will be:

\begin{align}
    \phi_{1} &= max(\bm{x_1}) \cdot w_{1} + b_{1} \\
    \phi_{2} &= max(\bm{x_2}) \cdot w_{2} + b_{2} \\
    \phi_{3} &= max(\bm{x_3}) \cdot w_{3} + b_{3} \\
    \phi_{4} &= max(\bm{x_4}) \cdot w_{4} + b_{4} \\
\end{align}


\begin{figure}[h]
    \centering
    \begin{quantikz}
        &&& \gate{R^1_x} &&& \\
        &&& \gate{R^2_x} &&& \\
        &&& \gate{R^3_x} &&& \\
        &&& \gate{R^4_x} &&& \\
    \end{quantikz}
    \caption{Pooling circuit for a 4 qubit architecture.}
    \label{fig:pooling-feature}
\end{figure}


Moreover, another fundamental concept of the \textit{block re-uploading} algorithm is that the 
number of qubits of the architecture corresponds to the number of blocks in which we divided the 
image, our main goal is to understand what is the optimal archi


\section{Numerical results}

We conducted several numerical test on the \textit{block re-uploading} architecture:

\begin{itemize}
    \item Classification.\\
    We conducted both binary and multi-class classification.
    \item Dataset.\\
    We used both the MNIST digit and MNIST fashion datasets, which are both grayscale $28\times28$ pixels images.
    \item Image size.\\
    We used different down-scaling of the MNIST dataset: $8\times8$, $12\times12$, $14\times14$, 
    $16\times16$, $18\times18$.
    \item Decoding observable.\\
    Every quantum machine learning algorithm has a \textit{decoding} component at the end of it, which consists in 
    measuring an observable to extract information from the PQC.
    The observable that we chose are: \textit{global Pauli Z}, which is the tensor product of n (number of qubits
    of the circuit) Pauli Z, \textit{local Pauli Z}, which is only one Pauli Z.
    Regarding the local Pauli Z measurement, in the block re-uploading architecture with multiple qubits, 
    we had the option to measure any qubit. However, we consistently chose to measure only the first qubit.
    \item Architectures.\\
    We studied three different architectures.
    The first one has only the embedding component, the second one has both the embedding and pooling components, the third
    one has the embedding, entanglement and pooling components.
    These architectures differ in terms of the number of gates and parameters they utilize, as shown by
    figure  
\end{itemize}

\begin{figure}[h]
    \centering
    \includegraphics[scale=0.7]{sections/chapters/Chapter6/Images/Architectures/embedding.pdf}
    \caption{This is the baseline model. This model only re-uploads multiple times the data in the circuit.}
    \label{arc:embed}
\end{figure}

\begin{figure}[h]
    \centering
    \includegraphics[scale=0.7]{sections/chapters/Chapter6/Images/Architectures/embedding_pooling.pdf}
    \caption{This model has both the embedding and the pooling circuit.}
    \label{arc:embed-pooling}
\end{figure}

It is essential to keep in mind that by increasing the size of images the number of embedding gates will
increase, hence an embedding architecture (baseline model) for $4\times4$ images will be different 
from the embedding architecture for $8\times8$.
The comparison between the baseline model for different sizes of the images is shown in figure 
\ref{arc:embed-pooling}

\begin{figure}[h]
    \centering
    \includegraphics[scale=0.4]{sections/chapters/Chapter6/Images/Image-Size/Heatmap-Comparison-Arch.pdf}
    \caption{The first row displays the number of parametric gates (excluding entangling gates) for 
    different image sizes across two architectures: embedding and embedding + pooling.
    The second row displays the number of trainable parameters for 
    different image sizes across two architectures: embedding and embedding + pooling.}
    \label{arc:embed-pooling}
\end{figure}


The following sections will discuss the \textit{generalization} capabilities and the 
\textit{trainability} of the \textit{block re-uploading} architecture.
I will group all the architectures combination in three main categories: \textit{narrow-deep} 
(few qubits and numerous layers), \textit{wide-shallow} (many qubits and few layers), \textit{proportionate}
(every other architecture). 


\subsection{Local 8x8} \label{sssec:num1}

\begin{figure}[h]
    \centering
    \includegraphics[scale=0.35]{sections/chapters/Chapter6/Images/Heatmaps/Local-Heatmaps/Heatmap-Both-8x8-L.pdf}
    \caption{This heatmap shows the training and validation accuracy for architectures with 1-15 qubits and
    1-6 layers for the $8\times8$ down-scaled MNIST digits and fashion dataset using a global Pauli Z.}
    \label{fig:heatmap-8x8-L}
\end{figure}
By looking at figure \ref{fig:heatmap-8x8-L}, we can distinguish 
three different behaviours:

\begin{enumerate}
    \item \textbf{Deep-Narrow}.\\
    As the number of layers increases in narrow architectures (1, 2, or 3 qubits), their capacity to 
    generalize diminishes, leading to overfitting. This occurs because the increase in layers corresponds 
    to a rise in the number of trainable parameters (see figure \ref{arc:embed-pooling}).\\
    As a result, the architecture becomes overparameterized, allowing it to capture even minor 
    fluctuations in the training dataset, which reduces its ability to generalize effectively to new, 
    unseen data.
    \item \textbf{Shallow-Wide}.\\
    As the number of qubits (width) increases both training and validation accuracy of 
    single-layer architectures (shallow) drop drastically.
    As the width of the architecture increases, a greater degree of entanglement is required to 
    effectively distribute information across all qubits. However, shallow architectures lack 
    sufficient entangling gates to achieve this, resulting in an inability to fully capture and 
    “understand" the complete picture.\\
    However, by comparing training and validation accuracy, although both decrease as 
    the width increases, there is no evidence of overfitting.\\
    We can conclude that, as the architecture widens, entanglement becomes increasingly crucial for the 
    architecture to effectively ”understand” the image.
    \item \textbf{Proportionate}.\\
    In proportionate architectures (those that are neither shallow-wide nor deep-narrow) overfitting 
    tends to diminish. As the number of layers and qubits increases, the \textit{block re-uploading} 
    introduces a richer set of entanglement structures, enabling more effective information sharing 
    across all qubits. This reduction in overfitting as entanglement grows raises some natural questions:
    \textit{Could entanglement be a factor that resists overfitting?}, 
    \textit{Might entanglement serve as a source of regularization?}.
\end{enumerate}


\subsection{Global 8x8}

\begin{figure}[h]
    \centering
    \includegraphics[scale=0.35]{sections/chapters/Chapter6/Images/Heatmaps/Global-Heatmaps/Heatmap-Both-8x8-G.pdf}
    \caption{This heatmap shows the training and validation accuracy for architectures with 1-15 qubits and
    1-6 layers for the $8\times8$ down-scaled MNIST digits and fashion dataset using a global Pauli Z.}
    \label{fig:heatmap-8x8-G}
\end{figure}


By looking at figure \ref{fig:heatmap-8x8-G}, we can notice 
several interesting behaviours:

\begin{itemize}
    \item \textbf{Deep vs Shallow}: Global vs Local.\\
    Shallow architectures clearly outperform deep ones, with overfitting becoming more pronounced 
    as the number of layers (depth) increases.\\
    This behavior is particularly noteworthy and contrasts with the numerical results observed 
    using a local Pauli Z.
    \item \textbf{Deep-Narrow}: Global vs Local.\\
    As stated in the previous point, overfitting is way more pronounced with respect to the local case.\\
    \item \textbf{Shallow-Wide}: Global vs Local.\\
    Shallow-Wide architectures (1-layer architectures) show an interesting behaviour: as the number of 
    qubits increases both training and validation accuracy of single-layer architectures 
    do not drop drastically, but remain stable.\\
    The lack of sufficient entanglement to share information across all qubits is compensated by 
    measuring a global observable, which effectively captures the dispersed information throughout the 
    circuit.
    \item \textbf{Proportionate}: Global vs Local.\\
    Proportionate architectures (center and bottom-right area of the heatmap) do not exhibit overfitting. \\
    However, they generally perform worse than their local counterparts.
 \end{itemize}

\subsection{Local 12x12} \label{sssec:local-12}
\begin{figure}[h]
    \centering
    \includegraphics[scale=0.3]{sections/chapters/Chapter6/Images/Heatmaps/Local-Heatmaps/Heatmap-Both-12x12-L.pdf}
    \caption{This heatmap shows the training and validation accuracy for architectures 
    trained on the $12\times12$ down-scaled MNIST digits and fashion dataset using a local Pauli Z.}
    \label{fig:heatmap-12x12-L}
\end{figure}

By enlarging the images, we observe a consistent trend across all categories 
(deep-narrow, shallow-wide, proportionate): the training accuracy remains basically the same, whereas we 
can notice an increasing presence of overfitting.
We can observe that the general behaviour of the validation accuracy seen with $8\times8$ images is 
similarly observed with $12\times12$ images, but \textit{shifted to the left by one or two layers}\footnote[1]{We are comparing only the local case. 
By general behaviour we intend that as the architecture becomes deeper overfitting becomes more 
pronounced.}. This shift 
occurs because the increased number of parameters in each layer makes the model more prone to overfitting.\\

By looking at figure \ref{fig:heatmap-12x12-L}, we can distinguish 
three different behaviours:

\begin{itemize}
    \item \textbf{Deep-Narrow}: $8\times8$ vs $12\times12$.\\
    Overfitting is more pronounced for deep-narrow architectures.\\
    In both the $8\times8$ and $12\times12$ cases, overfitting worsens as the number of layers increases. \\
    Training accuracy remains stable for $8\times8$ images, but for $12\times12$ images, 
    it decreases with the addition of more layers.
    \item \textbf{Shallow-Wide}: $8\times8$ vs $12\times12$.\\
    In the $8\times8$ case we noticed the drastic drop for both the validation and training accuracy for 
    architectures wider than 8 qubits, so since for the $12\times12$ we did not train architectures wider than 
    8 qubits we cannot conclude that we would see the same drastic drop.
    However, since deep-narrow and proportionate architectures in the $12\times12$ case did not show 
    significant differences with the $8\times8$ case we can assume that even Shallow-Wide architectures will
    not show significant differences.\\
    The effect of distributing information across more qubits in shallow architectures is particularly 
    evident in the $12\times12$ case. For instance, when examining 2-layer architectures vertically, 
    it's clear that spreading the information across more qubits significantly enhances both training and 
    validation accuracy.
    \item \textbf{Proportionate}: $8\times8$ vs $12\times12$.\\
    Again overfitting tends to disappear. \\
 \end{itemize}

\subsection{Global 12x12}
\begin{figure}[h]
    \centering
    \includegraphics[scale=0.3]{sections/chapters/Chapter6/Images/Heatmaps/Global-Heatmaps/Heatmap-Both-12x12-G.pdf}
    \caption{This heatmap shows the training and validation accuracy for architectures with 1-15 qubits and
    1-6 layers for the $12\times12$ down-scaled MNIST digits and fashion dataset using a global Pauli Z.}
    \label{fig:heatmap-12x12-G}
\end{figure}

By looking at figure \ref{fig:heatmap-12x12-G}, we can distinguish 
three different behaviours:

\begin{itemize}
    \item \textbf{Deep-Narrow}: $8 \times 8$ vs $12 \times 12$.\\
    
    \item \textbf{Shallow-Wide}: $8 \times 8$ vs $12 \times 12$.\\
    
    \item \textbf{Proportionate}: $8 \times 8$ vs $12 \times 12$.\\
    
 \end{itemize}

\subsection{Local 14x14, 16x16, 18x18}
\begin{figure}[h]
    \centering
    \includegraphics[scale=0.3]{sections/chapters/Chapter6/Images/Heatmaps/Local-Heatmaps/Heatmap-Both-14x14-L.pdf}
    \caption{This heatmap shows the training and validation accuracy for architectures for the $14\times14$ down-scaled MNIST digits and fashion dataset using a local Pauli Z.}
    \label{fig:heatmap-14x14-L}
\end{figure}

By looking at figure \ref{fig:heatmap-14x14-L}, we can distinguish 
three different behaviours:

\subsection{Global 14x14}
\begin{figure}[h]
    \centering
    \includegraphics[scale=0.3]{sections/chapters/Chapter6/Images/Heatmaps/Global-Heatmaps/Heatmap-Digits-14x14-G.pdf}
    \caption{This heatmap shows the training and validation accuracy for the $14\times14$ 
    down-scaled MNIST digits dataset using a global Pauli Z.}
    \label{fig:heatmap-14x14-G}
\end{figure}


By enlarging the images to $14\times14$, we can again notice a \textit{shift to the left by 
one or two layers} of the validation accuracy with respect to the $12\times12$ local case (figure 
\ref{fig:heatmap-14x14-L}). Once again, this shift 
occurs because the increased number of parameters in each layer makes the model more prone to overfitting.
The training accuracy for the $14\times14$ is essentially the same for both the $8\times8$ and 
$12\times12$ cases.

The remarks regarding the \textit{Deep-Narrow}, \textit{Shallow-Wide}, and \textit{Proportionate} 
architectures are consistent with those made in Section \ref{sssec:local-12} for the $12\times12$ case.
By examining Figure \ref{fig:heatmap-14x14-L}, no new behavior is observed that hasn't already 
been discussed in the $8\times8$ and $12\times12$ comparisons.

\subsection{Comparing few-qubits architectures}
\begin{figure}[h]
    \centering
    \includegraphics[scale=0.3]{sections/chapters/Chapter6/Images/Heatmaps/Local-Heatmaps/Heatmap-Digits-Small-Arch-L-train.pdf}
    \caption{}
    \label{fig:heatmap-Small-train}
\end{figure}

\begin{figure}[h]
    \centering
    \includegraphics[scale=0.3]{sections/chapters/Chapter6/Images/Heatmaps/Local-Heatmaps/Heatmap-Digits-Small-Arch-L-val.pdf}
    \caption{}
    \label{fig:heatmap-Small-val}
\end{figure}


\subsection{Comparing single-layer architectures}
\begin{figure}[h]
    \centering
    \includegraphics[scale=0.5]{sections/chapters/Chapter6/Images/Single-Layer/single-layer-comparison-compact-local-8x8.pdf}
    \caption{This figure shows the training and validation loss and accuracy for single-layer architectures
    trained on the $8\times8$ down-scaled MNIST digits dataset using a \textbf{local} Pauli Z.}
    \label{fig:Single-loss-local}
\end{figure}

\begin{figure}[h]
    \centering
    \includegraphics[scale=0.5]{sections/chapters/Chapter6/Images/Single-Layer/single-layer-comparison-compact-global-8x8.pdf}
    \caption{This figure shows the training and validation loss and accuracy for single-layer architectures
    trained on the $8\times8$ down-scaled MNIST digits dataset using a \textbf{global} Pauli Z.}
    \label{fig:Single-loss-global}
\end{figure}


As observed in subsection \ref{sssec:num1} and illustrated by the first column of the heatmap 
\ref{fig:heatmap-8x8-L}, there is a notable decrease in accuracy as the architectures become wider.\\
This behavior is not observed in the global case: as shown in heatmap \ref{fig:heatmap-8x8-G}, 
the accuracy remains stable as the architectures become wider, without the same significant drop.
This difference between the local and global cases can be attributed to the \textit{entanglement dynamics} and 
the \textit{type of observable used for decoding}. 
In the local case, the lack of entanglement limits information sharing across qubits. 
In contrast, a global observable, which aggregates information from all qubits, mitigates the 
impact of limited information sharing in wider architectures.\\

This intriguing behavior warrants a detailed discussion. 
In this section, we will analyze the evolution of loss functions and accuracies over 200 epochs, 
as illustrated in figures \ref{fig:Single-loss-local} and \ref{fig:Single-loss-global}.

\begin{itemize}
    \item \textbf{Training Loss}.\\ 
    An examination of the global and local training losses reveals that the 1-qubit configuration 
    has the lowest loss, with losses increasing as the number of qubits 
    increases\footnote[1]{This observation holds true except for the global 11-qubit configuration, 
    which appears as an outlier. This anomaly could be addressed by conducting a statistical analysis 
    with multiple training runs using different initialization seeds.}.
    As the number of qubits increases, the architecture becomes more challenging to train, resulting 
    in a higher loss function plateau in performance.
    \item \textbf{Validation Loss}.\\ 
    For validation loss, the trend of loss decreasing with the number of qubits is consistent 
    in the local case but not in the global case, where the losses are mixed.
    \item \textbf{Training-Validation Accuracy}.\\
    The local accuracy exhibits numerical instability as the number of qubits increases. 
    Initially smooth, the accuracy becomes increasingly "stair-stepped" with more qubits, 
    and eventually levels off, becoming constant for 14-qubit and 15-qubit architectures. 
    This progression illustrates how numerical instability becomes more pronounced with a 
    larger number of qubits.\\
    We can assert that the local 9-qubits (or 8-qubits) architecture represents the threshold beyond which numerical 
    instability becomes more pronounced.\\
    It is noteworthy that \textit{numerical instability is not observed in the global case}. 
    A possible explanation is that a global observable aggregates more information, which may 
    help mitigate the numerical instability seen when measuring an individual qubit.
\end{itemize}









\subsection{Trainability}
\begin{figure}[h]
    \centering
    \begin{subfigure}[b]{0.6\textwidth}
        \includegraphics[width=0.8\textwidth]{sections/chapters/Chapter6/Images/Trainability/variance-qubits-layer-local-uniform.pdf}
    \caption{This figure shows}
    \end{subfigure}
    \begin{subfigure}[b]{0.6\textwidth}
        \includegraphics[width=0.8\textwidth]{sections/chapters/Chapter6/Images/Trainability/variance-qubits-layer-global-uniform.pdf}
    \caption{This figure shows}
    \end{subfigure}
    \label{fig:local-uniform}
\end{figure}





\begin{figure}[h]
    \centering
    \includegraphics[width=\textwidth]{sections/chapters/Chapter6/Images/Trainability/variance-qubits-layer-local-uniform.pdf}
    \caption{This figure shows}
    \label{fig:local-gauss}
\end{figure}

\subsection{Conclusions}


Let's sum up the results found so far:

\begin{itemize}
    \item \textbf{Trainings: Deep-Narrow vs Shallow-Wide vs Proportionate (local).}\\
    We introduced three categories of architectures: Deep-Narrow, Shallow-Wide, and Proportionate. \\
    From our observations, three key patterns emerge: as Deep-Narrow architectures increase in depth, 
    overfitting becomes more pronounced; as Shallow-Wide architectures expand in width, both training and 
    validation accuracy decline, accompanied by the onset of numerical instabilities; and Proportionate 
    architectures, where information is evenly distributed across depth and width, tend to perform better overall.

    \item \textbf{Trainings: Deep-Narrow vs Shallow-Wide vs Proportionate (global).}\\
    Shallow-Wide architectures do not exhibit numerical instabilities and as the architectures
    becomes wider both training and validation accuracy remain stable (in contrast to the drop observed in 
    the local case).
    Shallow architectures (which can be narrow or wide) outperform (narrow or wide) deep architectures.

    \item \textbf{Trainings: Global vs Local.}\\
    Global decoding outperforms local decoding in shallow-wide architectures, as it avoids numerical instabilities. 
    However, for narrow-deep and proportionate architectures, local decoding proves to be more effective 
    than global decoding.

    \item \textbf{Trainings: Increasing size.}\\
    As the image size increases, we observe two distinct behaviors: training accuracy remains stable, 
    while validation accuracy decreases, indicating signs of overfitting. We describe this decrease 
    in validation accuracy as a \textit{shift to the left by one or two layers}. In other words, 
    as we increase the image size, the validation accuracy heatmap for larger images mirrors that of smaller 
    images, but shifted left by 1-2 layers.

    \item \textbf{Trainability: Global vs Local.} (uniform initialization)\\
    It is noteworthy that both decoding strategies exhibit the same behavior in 
    \textit{$Var(|\nabla_{\bm{\theta}}J|)$ vs qubits} and \textit{$Var(|\nabla_{\bm{\theta}}J|)$ vs layers}.
    The only distinction is that in \textit{$Var(|\nabla_{\bm{\theta}}J|)$ vs layers}, the global decoding 
    reaches the variance plateau much faster than the local decoding. Consequently, architectures with 
    local decoding have a larger layer budget at their disposal compared to those with global decoding.


\end{itemize}

\section{Outlook}

The future investigations of the \textit{block re-uploading} will be numerous:

\begin{itemize}
    \item \textbf{Architectures}.\\
    Thus far we have mostly investigated the following architecture:

    \begin{figure}[h]
        \centering
        \includegraphics[scale=0.65]{sections/chapters/Chapter6/Images/Two-Layers-A/two_layer.pdf}
        \caption{A four qubits and two layers \textit{block re-uploading} architecture. Each layer has three 
        components: an embedding circuit, an entanglement structure and a pooling circuit.
        Layers are separated by another entanglement structure.}
    \end{figure}

    In the future, we plan to explore simpler architectures before advancing to the more 
    complex \textit{block re-uploading} architecture. 
    Our plan will be a bottom-up approach, where we will \textit{progressively add new features} to the baseline model, 
    which has only the \textit{embedding} component (figure).
    Therefore, after studying the baseline model, we will explore the architectures with the 
    \textit{embedding} and \textit{pooling} components and finally we will study the architectures 
    with the \textit{embedding}, \textit{pooling} and \textit{entanglement} components.
    This step-by-step investigation will 
    help us assess the effectiveness of each component added to the baseline \textit{embedding} model.

    \begin{figure}[h]
        \centering
        \begin{subfigure}[b]{\textwidth}
            \centering
            \includegraphics[scale=0.58]{sections/chapters/Chapter6/Images/Architectures/embedding.pdf}
        \caption*{Baseline model.}
        \end{subfigure}
        \begin{subfigure}[b]{\textwidth}
            \centering
            \includegraphics[scale=0.62]{sections/chapters/Chapter6/Images/Architectures/embedding_pooling.pdf}
        \caption*{Baseline model with pooling feature.}
        \end{subfigure}
        \begin{subfigure}[b]{\textwidth}
            \centering
            \includegraphics[scale=0.68]{sections/chapters/Chapter6/Images/Two-Layers-B/two_layers_B.pdf}
        \caption*{Baseline model with pooling and entanglement features.}
        \end{subfigure}
    \end{figure}

    \item \textbf{Initialization}.\\
    Another interesting line of research could be to test different parameters initialization (thus far
    we tested only gaussian and uniform) to increase the architectures' \textit{layers budget}.
    
    \item \textbf{Warm-Up start}.\\
    So far, we have trained each architecture from scratch. In future experiments, we plan to explore 
    a new training strategy where an architecture with $L-1$ layers is first trained, and then, 
    for training an architecture with $L$ layers, we initialize the parameters of the first $L-1$ 
    layers using the trained parameters from the $L-1$ layer architecture.\\
    This a technique of \textit{transfer learning}.

    \item \textbf{Are shallow architectures contained in deep architectures?}.\\
    We would like to answer to this question for three different quantum re-uploading models:
    quantum re-uploading models where data and parameters are embedded in the same gates\footnote[1]{The 
    \textit{block re-uploading} architecture belongs to this first category}, 
    quantum re-uploading models where data and parameters are separated and the general 
    case\footnote[2]{In this final case, we aim to demonstrate that deep architectures inherently 
    include shallow architectures, even when considering architectures with varying numbers of qubits.}.\\
    Let's discuss these three categories separately.

    Regarding the first category, we can discuss it by considering the \textit{block re-uploading} 
    architecture.
    If we consider a two-layers \textit{block re-uploading} architecture, it is clear that by simply 
    initializing to zero the trainable parameters of the second layer we obtain a one-layer 
    \textit{block re-uploading} architecture.
    Therefore, every function that a one-layer \textit{block re-uploading} architecture can express can be 
    expressed by a two-layers \textit{block re-uploading} architecture.\\
    This opens up the question: \textit{Is it always true that deeper architectures contain 
    shallower ones?}\\
    
    If we consider a different data re-uploading approach where data and trainable parameters are 
    embedded in separate gates (second category) (figure \ref{fig:separate-data-var}), 
    determining whether deeper architectures inherently contain shallower ones becomes more complex.

    \begin{figure}[h]
        \centering
        \includegraphics[scale=0.7]{sections/chapters/Chapter6/Images/Separate-Data-Parameters/separate_data_param.pdf}
    \caption{This data re-uploading model has the variational component separated from the data embedding one.}
    \label{fig:separate-data-var}
    \end{figure}

    We could prove that a 2-layers architecture contains a 1-layer by adjusting the parameters $\theta_2$, 
    such that the quantum state $|\psi \rangle$ output from the first layer becomes an eigenvector 
    of the second layer.\\
    The crucial observation is that the data-embedding gates of the second layer, $U_2(x)$, 
    will first act on $|\psi \rangle$. Therefore, we need to adjust the parameters $\theta_2$ 
    to account for the effect of $U_2(x)$.\\

    Another approach to demonstrating that deep architectures contain shallower ones, when the data 
    embedding and variational components are separate, involves using \textit{truncated Fourier series}. 
    According to \cite{Schuld_2021}, quantum models can be represented as truncated Fourier series. 
    Thus, we could explore whether deep architectures contain shallower ones by showing that the 
    truncated Fourier series associated with a deep architecture includes more terms than the series 
    associated with a shallower architecture.\\

    Finally for the general case we could use \textit{tensor networks} to prove that deep 
    architectures contain shallower ones even if we consider architectures with varying number of 
    qubits.\\
    Every quantum circuit can be mapped to a tensor network\footnote[3]{While this is 
    always true, the reverse is not: the statement \textit{every tensor network is a quantum 
    circuit} is false. This is because quantum circuits impose specific constraints on tensor 
    networks, such as requiring the gates to be unitary matrices.}. Therefore, we can leverage 
    certain properties of tensor networks (figure \ref{fig:tensor-prop}) to transform one architecture with a fixed number of 
    qubits and layers into another architecture with a different number of qubits and layers.

    \begin{figure}
        \centering
        \includegraphics[scale=0.85]{sections/chapters/Chapter6/Images/Tensor-Networks/TN-properties.pdf}
        \caption{Some useful properties of tensor networks to transform a quantum circuit with a certain 
        topology into a quantum circuit with a different topology.}
        \label{fig:tensor-prop}
    \end{figure}

    For example, consider a 1-qubit, 4-layer circuit (see figure \ref{fig:1q-4l}). By leveraging 
    the properties of tensor networks, we can transform this circuit into a 2-qubit, 4-layer circuit 
    (see figure \ref{fig:2q-4l}). Further, we can bend it once more to obtain a 4-qubit, 
    1-layer circuit (see figure \ref{fig:4q-1l}). This heuristic argument suggests a potential path 
    for demonstrating that \textit{a deep circuit is equivalent to a wide circuit}. 
    More broadly, it implies a relationship between the depth and width of a 
    circuit\footnote[1]{Figures \ref{fig:1q-4l}, \ref{fig:2q-4l}, and \ref{fig:4q-1l} indicate 
    that an architecture with a certain depth $d$ can be equivalent to an architecture with width 
    $w = d$.}.\\

    \begin{figure}[h]
        \centering
        \begin{subfigure}[b]{\textwidth}
            \includegraphics[scale=0.5]{sections/chapters/Chapter6/Images/Tensor-Networks/1qubit4layers.pdf}
        \caption{This is a 1-qubit 4-layers quantum circuit. The blue circle and green circle represents two
        different components of a single layer. The red triangle represents the input state, whereas the 
        gray semi-circle represents the measurement.}
        \label{fig:1q-4l}
        \end{subfigure}
        \\[3ex]
        \begin{subfigure}[b]{\textwidth}
        \centering
            \includegraphics[scale=0.5]{sections/chapters/Chapter6/Images/Tensor-Networks/2layers2qubits.pdf}
        \caption{By bending the 1-qubit 4-layers quantum circuit, we obtain a 2-qubits 2-layers circuit. If 
        we use tensor networks' properties show in figure \ref{fig:tensor-prop} we obtain a 2-qubits 
        2-layers circuit.}
        \label{fig:2q-2l}
        \end{subfigure}
        \\[3ex]
        \begin{subfigure}[b]{\textwidth}
        \centering
            \includegraphics[scale=0.5]{sections/chapters/Chapter6/Images/Tensor-Networks/4qubits1layer.pdf}
        \caption{By bending the 2-qubit 2-layers quantum circuit, we obtain a 4-qubits 1-layer circuit.}
        \label{fig:4q-1l}
        \end{subfigure}
        \caption{This figure shows how we can transform a 1-qubit 4-layers architecture in 4-qubits 1-layer 
        circuit by exploiting tensor networks' properties.}
    \end{figure}

    


\end{itemize}

