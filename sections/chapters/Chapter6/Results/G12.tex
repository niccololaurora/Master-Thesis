\begin{figure}[h]
    \centering
    \includegraphics[scale=0.3]{sections/chapters/Chapter6/Images/Heatmaps/Global-Heatmaps/Heatmap-Both-12x12-G.pdf}
    \caption{This heatmap shows the training and validation accuracy for the $12\times12$ 
    down-scaled MNIST digits and fashion dataset using a global Pauli Z.}
    \label{fig:heatmap-12x12-G}
\end{figure}

By enlarging the images, we observe a consistent trend across all categories 
(deep-narrow, shallow-wide, proportionate): the validation accuracy \textit{systematically decreases}, 
wherease the training accuracy remains \textit{stable}.
Moreover, overfitting occurs even in 2-layers architectures (evident in the 3-qubit, 2-layer models), 
which is surprising since overfitting at such low depths was not observed with $8\times8$ images.
We can observe that the behaviour of the validation accuracy seen with $8\times8$ images is similarly 
observed with $12\times12$ images, but \textit{shifted to the left by one or two layers}. 
This shift occurs because the increased number of parameters in each layer makes the model more prone to 
overfitting.\\
Furthermore, when comparing the global and local cases for $12\times12$ images, the global 
case clearly performs worse in both training and validation accuracy.\\
By looking at figure \ref{fig:heatmap-12x12-G}, we can distinguish 
three different behaviours:

\begin{itemize}
    \item \textbf{Deep vs Shallow}: $8\times8$ vs $12\times12$.\\
    It is evident that deep architectures lose the generalization capabilities of shallow architectures:
    almost every architecture with 3 or more layers shows signs of overfitting.
    In the $8\times8$ we have clear signs of overfitting only for architectures with 5 or 6 layers.
    \item \textbf{Deep-Narrow}: $8\times8$ vs $12\times12$.\\
    Once again, we observe that increasing the number of layers leads to overfitting. 
    Compared to the $8 \times 8$ case, the overfitting is more pronounced in this instance.
    \item \textbf{Shallow-Wide}: $8\times8$ vs $12\times12$.\\
    There are no substantial differences between the $8\times8$ and $12\times12$ scenarios.
    \item \textbf{Proportionate}: $8\times8$ vs $12\times12$.\\
    In proportionate architectures, the shifting behavior of overfitting and accuracy 
    is evident: the drop in accuracy that occurs as the number of layers increases in the $8 \times 8$ 
    images appears a few layers earlier in the $12 \times 12$ scenario.
 \end{itemize}