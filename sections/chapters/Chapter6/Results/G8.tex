
\begin{figure}[h]
    \centering
    \includegraphics[scale=0.3]{sections/chapters/Chapter6/Images/Heatmaps/Global-Heatmaps/Heatmap-Both-8x8-G.pdf}
    \caption{This heatmap shows the training and validation accuracy for architectures with 1-15 qubits and
    1-6 layers for the $8\times8$ down-scaled MNIST digits and fashion dataset using a global Pauli Z.}
    \label{fig:heatmap-8x8-G}
\end{figure}


By looking at figure \ref{fig:heatmap-8x8-G}, we can notice 
several interesting behaviours:

\begin{itemize}
    \item \textbf{Deep vs Shallow}: Global vs Local.\\
    Shallow architectures clearly outperform deep ones, with overfitting becoming more pronounced 
    as the number of layers (depth) increases.\\
    This behavior is particularly noteworthy and contrasts with the numerical results observed 
    using a local Pauli Z.
    \item \textbf{Deep-Narrow}: Global vs Local.\\
    As stated in the previous point, overfitting is way more pronounced with respect to the local case.
    \item \textbf{Shallow-Wide}: Global vs Local.\\
    Shallow-Wide architectures (1-layer architectures) show an interesting behaviour: as the number of 
    qubits increases both training and validation accuracy of single-layer architectures 
    do not drop drastically, but remain stable.\\
    The lack of sufficient entanglement to share information across all qubits is compensated by 
    measuring a global observable, which effectively captures the dispersed information throughout the 
    circuit.
    \item \textbf{Proportionate}: Global vs Local.\\
    Proportionate architectures (center and bottom-right area of the heatmap) do not exhibit overfitting. \\
    However, they generally perform worse than their local counterparts.
 \end{itemize}