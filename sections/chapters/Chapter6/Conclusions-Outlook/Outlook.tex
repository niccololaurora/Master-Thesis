
The future investigations of the \textit{block re-uploading} will be numerous:

\begin{itemize}
    \item \textbf{Architectures}.\\
    Thus far we have mostly investigated the following architecture:

    \begin{figure}[h]
        \centering
        \includegraphics[scale=0.65]{sections/chapters/Chapter6/Images/Two-Layers-A/two_layer.pdf}
        \caption{A four qubits and two layers \textit{block re-uploading} architecture. Each layer has three 
        components: an embedding circuit, an entanglement structure and a pooling circuit.
        Layers are separated by another entanglement structure.}
    \end{figure}

    In the future, we plan to explore simpler architectures before advancing to the more 
    complex \textit{block re-uploading} architecture. 
    Our plan will be a bottom-up approach, where we will \textit{progressively add new features} to the baseline model, 
    which has only the \textit{embedding} component (figure).
    Therefore, after studying the baseline model, we will explore the architectures with the 
    \textit{embedding} and \textit{pooling} components and finally we will study the architectures 
    with the \textit{embedding}, \textit{pooling} and \textit{entanglement} components.
    This step-by-step investigation will 
    help us assess the effectiveness of each component added to the baseline \textit{embedding} model.

    \begin{figure}[h]
        \centering
        \begin{subfigure}[b]{\textwidth}
            \centering
            \includegraphics[scale=0.58]{sections/chapters/Chapter6/Images/Architectures/embedding.pdf}
        \caption*{Baseline model.}
        \end{subfigure}
        \begin{subfigure}[b]{\textwidth}
            \centering
            \includegraphics[scale=0.62]{sections/chapters/Chapter6/Images/Architectures/embedding_pooling.pdf}
        \caption*{Baseline model with pooling feature.}
        \end{subfigure}
        \begin{subfigure}[b]{\textwidth}
            \centering
            \includegraphics[scale=0.68]{sections/chapters/Chapter6/Images/Two-Layers-B/two_layers_B.pdf}
        \caption*{Baseline model with pooling and entanglement features.}
        \end{subfigure}
    \end{figure}

    \item \textbf{Initialization}.\\
    Another interesting line of research could be to test different parameters initialization (thus far
    we tested only gaussian and uniform) to increase the architectures' \textit{layers budget}.
    
    \item \textbf{Warm-Up start}.\\
    So far, we have trained each architecture from scratch. In future experiments, we plan to explore 
    a new training strategy where an architecture with $L-1$ layers is first trained, and then, 
    for training an architecture with $L$ layers, we initialize the parameters of the first $L-1$ 
    layers using the trained parameters from the $L-1$ layer architecture.\\
    This a technique of \textit{transfer learning}.

    \item \textbf{Are shallow architectures contained in deep architectures?}.\\
    We would like to answer to this question for three different quantum re-uploading models:
    quantum re-uploading models where data and parameters are embedded in the same gates\footnote[1]{The 
    \textit{block re-uploading} architecture belongs to this first category}, 
    quantum re-uploading models where data and parameters are separated and the general 
    case\footnote[2]{In this final case, we aim to demonstrate that deep architectures inherently 
    include shallow architectures, even when considering architectures with varying numbers of qubits.}.\\
    Let's discuss these three categories separately.

    Regarding the first category, we can discuss it by considering the \textit{block re-uploading} 
    architecture.
    If we consider a two-layers \textit{block re-uploading} architecture, it is clear that by simply 
    initializing to zero the trainable parameters of the second layer we obtain a one-layer 
    \textit{block re-uploading} architecture.
    Therefore, every function that a one-layer \textit{block re-uploading} architecture can express can be 
    expressed by a two-layers \textit{block re-uploading} architecture.\\
    This opens up the question: \textit{Is it always true that deeper architectures contain 
    shallower ones?}\\
    
    If we consider a different data re-uploading approach where data and trainable parameters are 
    embedded in separate gates (second category) (figure \ref{fig:separate-data-var}), 
    determining whether deeper architectures inherently contain shallower ones becomes more complex.

    \begin{figure}[h]
        \centering
        \includegraphics[scale=0.7]{sections/chapters/Chapter6/Images/Separate-Data-Parameters/separate_data_param.pdf}
    \caption{This data re-uploading model has the variational component separated from the data embedding one.}
    \label{fig:separate-data-var}
    \end{figure}

    We could prove that a 2-layers architecture contains a 1-layer by adjusting the parameters $\theta_2$, 
    such that the quantum state $|\psi \rangle$ output from the first layer becomes an eigenvector 
    of the second layer.\\
    The crucial observation is that the data-embedding gates of the second layer, $U_2(x)$, 
    will first act on $|\psi \rangle$. Therefore, we need to adjust the parameters $\theta_2$ 
    to account for the effect of $U_2(x)$.\\

    Another approach to demonstrating that deep architectures contain shallower ones, when the data 
    embedding and variational components are separate, involves using \textit{truncated Fourier series}. 
    According to \cite{Schuld_2021}, quantum models can be represented as truncated Fourier series. 
    Thus, we could explore whether deep architectures contain shallower ones by showing that the 
    truncated Fourier series associated with a deep architecture includes more terms than the series 
    associated with a shallower architecture.\\

    Finally for the general case we could use \textit{tensor networks} to prove that deep 
    architectures contain shallower ones even if we consider architectures with varying number of 
    qubits.\\
    Every quantum circuit can be mapped to a tensor network\footnote[3]{While this is 
    always true, the reverse is not: the statement \textit{every tensor network is a quantum 
    circuit} is false. This is because quantum circuits impose specific constraints on tensor 
    networks, such as requiring the gates to be unitary matrices.}. Therefore, we can leverage 
    certain properties of tensor networks to transform one architecture with a fixed number of 
    qubits and layers into another architecture with a different number of qubits and layers.

    For example, consider a 1-qubit, 4-layer circuit (see Figure \ref{fig:1q-4l}). By leveraging 
    the properties of tensor networks, we can transform this circuit into a 2-qubit, 4-layer circuit 
    (see figure \ref{fig:2q-4l}). Further, we can bend it once more to obtain a 4-qubit, 
    1-layer circuit (see Figure \ref{fig:4q-1l}). This heuristic argument suggests a potential path 
    for demonstrating that \textit{a deep circuit is equivalent to a wide circuit}. 
    More broadly, it implies a relationship between the depth and width of a 
    circuit\footnote[1]{Figures \ref{fig:1q-4l}, \ref{fig:2q-4l}, and \ref{fig:4q-1l} indicate 
    that an architecture with a certain depth $d$ can be equivalent to an architecture with width 
    $w = d$.}.
    
    
    an equivalence between layers and qubits: 
    $\# qubits = \# layers$.
    However, if we want to come back to the proof of the initial statement that \textit{deep architectures 
    contain shallower ones} by freeing-up the state preparation (which is the red triangle in the 
    diagrams) and measurement (gray semicircle in the diagrams) 

    \begin{figure}[h]
        \centering
        \begin{subfigure}[b]{\textwidth}
            \includegraphics[scale=0.5]{sections/chapters/Chapter6/Images/Tensor-Networks/1qubit4layers.pdf}
        \caption{This is a 1-qubit 4-layers quantum circuit. The blue circle and green circle represents two
        different components of a single layer. The red triangle represents the input state, whereas the 
        gray semi-circle represents the measurement.}
        \label{fig:1q-4l}
        \end{subfigure}
        \begin{subfigure}[b]{\textwidth}
        \vspace{10pt}
        \centering
            \includegraphics[scale=0.5]{sections/chapters/Chapter6/Images/Tensor-Networks/2layers2qubits.pdf}
        \caption{By bending the 1-qubit 4-layers quantum circuit, we obtain a 2-qubits 2-layers circuit.}
        \label{fig:2q-2l}
        \end{subfigure}
        \begin{subfigure}[b]{\textwidth}
        \vspace{10pt}
        \centering
            \includegraphics[scale=0.5]{sections/chapters/Chapter6/Images/Tensor-Networks/4qubits1layer.pdf}
        \caption{By bending the 2-qubit 2-layers quantum circuit, we obtain a 4-qubits 1-layer circuit.}
        \label{fig:4q-1l}
        \end{subfigure}
        \caption{This figure show how we can transform a 1-qubit 4-layers architecture in 4-qubits 1-layer 
        circuit by exploiting tensor networks' properties.}
    \end{figure}

    


\end{itemize}