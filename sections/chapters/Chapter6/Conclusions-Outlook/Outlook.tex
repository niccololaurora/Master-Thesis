
The future investigations of the \textit{block re-uploading} will be numerous:

\begin{itemize}
    \item \textbf{Architectures}: progressive addition of features.\\
    Thus far we have mostly investigated the following architecture:

    \begin{figure}[h]
        \centering
        \includegraphics[scale=0.65]{sections/chapters/Chapter6/Images/Two-Layers-A/two_layer.pdf}
        \caption{A four qubits and two layers \textit{block re-uploading} architecture. Each layer has three 
        components: an embedding circuit, an entanglement structure and a pooling circuit.
        Layers are separated by another entanglement structure.}
    \end{figure}

    In the future, we plan to explore simpler architectures before advancing to the more 
    complex \textit{block re-uploading} architecture. 
    Our plan will be bottom-up approach, where we will progressively add new features to the baseline model, 
    which has only the \textit{embedding} component (figure).
    Therefore, after studying the baseline model, we will explore the architectures with the 
    \textit{embedding} and \textit{pooling} components and finally we will study the architectures 
    with the \textit{embedding}, \textit{pooling} and \textit{entanglement} components.
    This step-by-step investigation will 
    help us assess the effectiveness of each component added to the baseline \textit{embedding} model.

    \begin{figure}[h]
        \centering
        \begin{subfigure}[b]{\textwidth}
            \includegraphics[scale=0.5]{sections/chapters/Chapter6/Images/Architectures/embedding.pdf}
        \caption{Baseline model.}
        \end{subfigure}
        \begin{subfigure}[b]{\textwidth}
            \includegraphics[scale=0.6]{sections/chapters/Chapter6/Images/Architectures/embedding_pooling.pdf}
        \caption{Baseline model with the pooling feature.}
        \end{subfigure}
        \begin{subfigure}[b]{\textwidth}
            \includegraphics[scale=0.7]{sections/chapters/Chapter6/Images/Two-Layers-B/two_layers_B.pdf}
        \caption{Baseline model with the pooling and entanglement features.}
        \end{subfigure}
    \end{figure}

\end{itemize}